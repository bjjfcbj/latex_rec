\documentclass[UTF8]{ctexart}
\title{daydayup}
\author{drmin}
\date{\today}
\usepackage{ctex}
\usepackage{amsmath}
\usepackage{graphicx}
\usepackage{geometry}%页边距
\geometry{a4paper}
\geometry{left=1cm,right=2cm,top=3cm,bottom=4cm}
\usepackage{fancyhdr}%页眉页脚
\pagestyle{fancy}
\lhead{\author}
\chead{\date}
\rhead{222}
\lfoot{}
\cfoot{\thepage}
\rfoot{}
\renewcommand{\headrulewidth}{0.4pt}
\renewcommand{\headwidth}{\textwidth}
\renewcommand{\footrulewidth}{0pt}
\usepackage{setspace}%行间距
\onehalfspacing
\begin{document}
\maketitle
\newpage
\tableofcontents
\newpage
\section{exampleOFlatex}
latex这部分需要对着tex文件看源码和注释。
\subsection{math}
\subsubsection{上下标与公式环境}
行内$x=y$公式以及行间公式:\[x=y\]
行间公式编号(\^{} 为上标,\_{}为下标):
%要写成\^{}的形式,不然会与后一字符重叠
\begin{equation}
x_n=y^2
\end{equation}
\begin{equation}
y^{n1,n2,n3}=x_{n1,n2,n3}
\end{equation}
\subsubsection{运算符}
根式用\texttt{\char`\\}sqrt\{.\}表示,分式用\texttt{\char`\\}frac\{.\}\{.\}表示(前为分子,后为分母)
%\的输出就是这么麻烦,命令\texttt{}使用tt字体环境,\char`加字符表示输出ASCII字符,保留字符前应该添加反斜杠\
\begin{equation}
\sqrt{x}=\frac{y}{z}
\end{equation}
注意分式在行间行内显示效果不同(大小区别):$\frac{y}{z}$
	
然后是其他常用需要命令生成的运算符:\[ \pm\;\times\;\div\;\cdot\;\cap\;\cup\;\geq\;\leq\;\neq\;\approx\;\equiv\;\sum;\prod;\lim;\int \]
其中行内行间表现不同(可以用\texttt{\char`\\}limits以及\texttt{\char`\\}nolimits调整压缩):
	
$ \sum_{i=1}^n i\quad \prod_{i=1}^n $
$ \sum\limits _{i=1}^n i\quad \prod\limits _{i=1}^n $
\[ \lim_{x\to0}x^2 \quad \int_a^b x^2 dx \]
\[ \lim\nolimits _{x\to0}x^2\quad \int\nolimits_a^b x^2 dx \]
		
多重积分符号:\[ \iint\quad \iiint\quad \iiiint\quad \idotsint \]
		
\subsubsection{定界符,省略符,矩阵}
诸如(),[],\{\}之类的符号及大小控制:
\[ \Biggl(\biggl(\Bigl(\bigl((x)\bigr)\Bigr)\biggr)\Biggr) \]
\[ \Biggl[\biggl[\Bigl[\bigl[[x]\bigr]\Bigr]\biggr]\Biggr] \]
\[ \Biggl \{\biggl \{\Bigl \{\bigl \{\{x\}\bigr \}\Bigr \}\biggr \}\Biggr\} \]
\[ \Biggl\langle\biggl\langle\Bigl\langle\bigl\langle\langle x
\rangle\bigr\rangle\Bigr\rangle\biggr\rangle\Biggr\rangle \]
\[ \Biggl\lvert\biggl\lvert\Bigl\lvert\bigl\lvert\lvert x
\rvert\bigr\rvert\Bigr\rvert\biggr\rvert\Biggr\rvert \]
\[ \Biggl\lVert\biggl\lVert\Bigl\lVert\bigl\lVert\lVert x
\rVert\bigr\rVert\Bigr\rVert\biggr\rVert\Biggr\rVert \]
\[ x_1,x_2,\dots ,x_n\quad 1,2,\cdots ,n\quad \vdots\quad \ddots \]
\[ \begin{pmatrix} a&b\\c&d \end{pmatrix} \quad
\begin{bmatrix} a&b\\c&d \end{bmatrix} \quad
\begin{Bmatrix} a&b\\c&d \end{Bmatrix} \quad
\begin{vmatrix} a&b\\c&d \end{vmatrix} \quad
\begin{Vmatrix} a&b\\c&d \end{Vmatrix} \]
以及行内小矩阵$ ( \begin{smallmatrix} a&b\\c&d \end{smallmatrix} ) $.

\subsubsection{公式格式}
多行公式(非对齐):
\begin{multline}
	x=\\
	a+b+c+\\
	h+i+\\
	d+e+f+g
\end{multline}
对齐多行公式(必须在数学环境中):
\[\begin{aligned}
	x ={}& a+b+c+{} \\
	&d+e+f+g
\end{aligned}\]
公式组:
\begin{gather}
	a = b+c+d \\
	x = y+z
\end{gather}
\begin{align}
	a &= b+c+d \\
	x &= y+z
\end{align}
分段函数:
\[ y= \begin{cases}
	-x,\quad x\leq 0 \\
	x,\quad x>0
\end{cases} \]
\subsection{图片,表格}
\subsubsection{图片}
\includegraphics[width=.5\linewidth]{timg.jpg}

\subsubsection{表格}
\begin{tabular}{|l|c|r|}
	\hline
	操作系统& 发行版& 编辑器\\
	\hline
	Windows & MikTeX & TexMakerX \\
	\hline
	Unix/Linux & teTeX & Kile \\
	\hline
	Mac OS & MacTeX & TeXShop \\
	\hline
	通用& TeX Live & TeXworks \\
	\hline
\end{tabular}
\begin{tabular}{|c|r|c|r|}
	\hline
	latex命令& 输出& latex命令& 输出\\
	\hline
	\textbackslash\# & \# & \textbackslash\$ & \$ \\
	\hline
	\textbackslash\% & \% & \textbackslash\{ & \{ \\
	\hline
	\textbackslash\} & \} & \textbackslash\~{}\{\} & \~{} \\
	\hline
	\textbackslash\_{}\{\} & \_{} & \textbackslash\^{}\{\} & \^{} \\
	\hline
	\textbackslash textbackslash & \textbackslash & \textbackslash\& & \& \\
	\hline
\end{tabular}

\subsubsection{浮动体}
\begin{figure}[htbp]
	\centering
	\includegraphics[width=.3\linewidth]{timg.jpg}
	\caption{有图有真相}
	\label{fig:myphoto}
\end{figure}

\end{document}